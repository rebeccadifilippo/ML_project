\documentclass{article}
\usepackage{graphicx}
\usepackage{tabularx}
\usepackage{booktabs}
\graphicspath{ {.} }

\title{Proposal and Contract for 4AL3 ML Project}
\author{Rebecca Di Filippo, Jake Read, Claire Nielsen}
\date{\today}

\begin{document}

\maketitle

\newpage
\section{Team Members}

\textbf{Rebecca Di Filippo}\\
difilir@mcmaster.ca\\

\textbf{Jake Read}\\
readj9@mcmaster.ca\\

\textbf{Claire Nielsen}\\
nielsc2@mcmaster.ca\\

\section{Task Title and Overview}

\section{Task Definition}
(Type of data, classification/regression/generation, number of classes, single-label or multi-label\\

\section{Problem Description}
Describe the problem, it’s impact on the real world, and why it is challenging \\

\section{Data Source(s) and Plan for Data Collection.}
This may include how you are going to scrape the data and follow terms-of-service, API access and handling rate limiting, using open-source data, or any other relevant details. If your data does not have labels, how do you plan to get them? If assigning labels by hand, how long does it take per instance? Include links to the data if relevant. If you download your dataset from Kaggle, you must include the Kaggle link and the original data source link from where the dataset was downloaded and posted on
Kaggle. Include any meta-data available for your corpus. If you are not collecting it yourself, include details about how the data was collected, annotated, preprocessed, etc. Do not work on the datasets for which no source information is available. Please indicate whether you will use a small subset of the data or features for your project or the entire dataset, and why. \\

\section{Expected Size of the Dataset}
(number of data points) and 3 example data points with labels. Your dataset should have at least 1k data points. Some projects will have more or less data. \\

\section{Proposed Solution}
How do you plan to go about solving this problem? It is okay to not know
how the machine learning models work at this point in the class, but you should be starting to get some idea based on the lectures and assigned readings. What kind of features and target labels do you have? What kind of models might you try? You may not propose simple linear regression models for this project. Are there any existing solutions to your problem?
You must indicate 2-5 sources (research papers, books, machine learning challenges online) from where your solution was inspired. Put some thought into how you would approach this. How will you know if the model is good? How will you evaluate it? Also, share the libraries you intend to use for the project. \\

\newpage
\section{Team Charter}
\subsection{Purpose}

Our teams purpose is to...\\

\subsection{Team Member Roles}

The team will rotate to fill several roles throughout the course of the project. The roles are: 

\begin{table}[ht]
\centering
\caption{Non-technical Roles and Descriptions}
\label{tab:non-technical-roles}
\begin{tabularx}{\textwidth}{lX}
\toprule
\textbf{Role} & \textbf{Description} \\
\midrule

Meeting Chair & Prepares meeting agenda or discussion points ahead of time. Facilitates team discussions during the meeting to ensure all required points are addressed. When required, leads delegation of workload.  \\ \\
Team Liaison & Acts as the main point of contact between the team and supervisors and/or stakeholders.  \\ \\
Note Taker & Records meeting minutes. \\ \\
\bottomrule
\end{tabularx}
\end{table}

The team will assign workload based on even delegation of work, as well as personal expertise or comfort in specific areas of the project. Each member will fill a technical role of developer. \\

\subsection{Leadership}

The team will delegate and divide workflow as a collaborative discussion. The Meeting Chair may direct the conversation but all members of the team hold equal decision making power. In the event of an impasse a vote will be held between members; in the event of a stalemate the team will seek advice from peers. \\

\subsection{Meeting Plan}

The team will meet once a week on Thursdays at 3:30pm. Depending on impending workload or deliverables, the team may choose to meet additional times, subject to members' availability. If any member has a conflict with the usual meeting slot, an alternate time may be chosen

\subsection{Communication Plan}

The team will communicate primarily using a discord chat. The chat will be used to communicate over text as well as meeting calls. Any correspondence with a TA should be done over Outlook or Teams. \\

All teammates will endeavor to attend every scheduled meeting. Should the team unanimously decide a meeting is not required, it may be canceled ahead of time. If a teammate is not able to attend a scheduled meeting, they must provide the others with notice in advance, as early as possible and up to twenty-four hours in advance barring emergency circumstances. Permissible emergency circumstances include but are not limited to: \\

\begin{itemize}
    \item Medical concern (migraine, illness, etc.)
    \item Family emergency
    \item Unforseen transit issues \\
\end{itemize}

The same list of circumstances applies should a teammate be responsible for the team missing a deliverable deadline. If a deadline is missed, the team will discuss how to avoid the situation by next scheduled meeting latest. \\

\textbf{Team Charter Signatures}\\

image here\\
Rebecca Di Filippo \\ 
\date{\today}\\ \\ \\ 

image here\\
Jake Read\\
\date{\today }\\ \\ \\ 

\includegraphics[width=0.3\textwidth]{signature_claire}\\
Claire Nielsen\\
\date{\today}


\end{document}