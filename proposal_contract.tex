\documentclass{article}

\title{Proposal and Contract for 4AL3 ML Project}
\author{Rebecca Di Filippo, Jake Read, Claire Neilsen}
\date{\today}

\begin{document}

\maketitle

\section{Team Members}
% Already listed in the author field

\section{Task Title and Overview}
\textbf{Task Title:} Asteroid Classification Using Machine Learning

\textbf{Overview:} 
This project aims to classify asteroids based on their orbital and physical characteristics to predict potentially hazardous asteroids (PHAs) versus non-hazardous ones. 
Accurate classification of asteroids is crucial for early warning systems, space mission planning, and planetary defense.

\textbf{Significance:}
\begin{itemize}
    \item Helps prevent potential asteroid impacts on Earth.
    \item Provides insights into asteroid characteristics for scientific and space exploration purposes.
\end{itemize}

\textbf{Challenges:}
\begin{itemize}
    \item Class imbalance: less hazardous asteroids in the set, which may affect model performance.
    \item Some features may be correlated or redundant which requires feature selection techniques.
\end{itemize}

\section{Task Definition}
\begin{itemize}
    \item \textbf{Type of Data:} Structured dataset with numerical features (e.g., orbital parameters, size) and categorical labels (hazardous vs. non-hazardous).
    \item \textbf{Task Type:} Supervised classification.
    \item \textbf{Number of Classes:} 2 (Potentially Hazardous, Not Hazardous)
    \item \textbf{Label Type:} Single-label (asteroid is either hazardous or not).
\end{itemize}

\section{Problem Description}
Asteroids pose a threat to Earth.
While most asteroids orbit safely, some come close enough to be considered potentially hazardous and could cause significant damage if they were to impact Earth.
Predicting which asteroids are hazardous allows space agencies to prioritize observation on certain asteroids, develop mitigation strategies, and design early-warning systems.


\end{document}
